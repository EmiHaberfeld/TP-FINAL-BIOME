% Options for packages loaded elsewhere
\PassOptionsToPackage{unicode}{hyperref}
\PassOptionsToPackage{hyphens}{url}
%
\documentclass[
]{article}
\title{Trabajo Práctico Final\\
Biometría II - 2021}
\author{Matías Alemán, Milagros Azcueta, Manuel Fiz, Emilia Haberfeld,
Diego Kafer, Ilan Shalom}
\date{21/10/2021}

\usepackage{amsmath,amssymb}
\usepackage{lmodern}
\usepackage{iftex}
\ifPDFTeX
  \usepackage[T1]{fontenc}
  \usepackage[utf8]{inputenc}
  \usepackage{textcomp} % provide euro and other symbols
\else % if luatex or xetex
  \usepackage{unicode-math}
  \defaultfontfeatures{Scale=MatchLowercase}
  \defaultfontfeatures[\rmfamily]{Ligatures=TeX,Scale=1}
\fi
% Use upquote if available, for straight quotes in verbatim environments
\IfFileExists{upquote.sty}{\usepackage{upquote}}{}
\IfFileExists{microtype.sty}{% use microtype if available
  \usepackage[]{microtype}
  \UseMicrotypeSet[protrusion]{basicmath} % disable protrusion for tt fonts
}{}
\makeatletter
\@ifundefined{KOMAClassName}{% if non-KOMA class
  \IfFileExists{parskip.sty}{%
    \usepackage{parskip}
  }{% else
    \setlength{\parindent}{0pt}
    \setlength{\parskip}{6pt plus 2pt minus 1pt}}
}{% if KOMA class
  \KOMAoptions{parskip=half}}
\makeatother
\usepackage{xcolor}
\IfFileExists{xurl.sty}{\usepackage{xurl}}{} % add URL line breaks if available
\IfFileExists{bookmark.sty}{\usepackage{bookmark}}{\usepackage{hyperref}}
\hypersetup{
  pdftitle={Trabajo Práctico Final Biometría II - 2021},
  pdfauthor={Matías Alemán, Milagros Azcueta, Manuel Fiz, Emilia Haberfeld, Diego Kafer, Ilan Shalom},
  hidelinks,
  pdfcreator={LaTeX via pandoc}}
\urlstyle{same} % disable monospaced font for URLs
\usepackage[margin=1in]{geometry}
\usepackage{color}
\usepackage{fancyvrb}
\newcommand{\VerbBar}{|}
\newcommand{\VERB}{\Verb[commandchars=\\\{\}]}
\DefineVerbatimEnvironment{Highlighting}{Verbatim}{commandchars=\\\{\}}
% Add ',fontsize=\small' for more characters per line
\usepackage{framed}
\definecolor{shadecolor}{RGB}{248,248,248}
\newenvironment{Shaded}{\begin{snugshade}}{\end{snugshade}}
\newcommand{\AlertTok}[1]{\textcolor[rgb]{0.94,0.16,0.16}{#1}}
\newcommand{\AnnotationTok}[1]{\textcolor[rgb]{0.56,0.35,0.01}{\textbf{\textit{#1}}}}
\newcommand{\AttributeTok}[1]{\textcolor[rgb]{0.77,0.63,0.00}{#1}}
\newcommand{\BaseNTok}[1]{\textcolor[rgb]{0.00,0.00,0.81}{#1}}
\newcommand{\BuiltInTok}[1]{#1}
\newcommand{\CharTok}[1]{\textcolor[rgb]{0.31,0.60,0.02}{#1}}
\newcommand{\CommentTok}[1]{\textcolor[rgb]{0.56,0.35,0.01}{\textit{#1}}}
\newcommand{\CommentVarTok}[1]{\textcolor[rgb]{0.56,0.35,0.01}{\textbf{\textit{#1}}}}
\newcommand{\ConstantTok}[1]{\textcolor[rgb]{0.00,0.00,0.00}{#1}}
\newcommand{\ControlFlowTok}[1]{\textcolor[rgb]{0.13,0.29,0.53}{\textbf{#1}}}
\newcommand{\DataTypeTok}[1]{\textcolor[rgb]{0.13,0.29,0.53}{#1}}
\newcommand{\DecValTok}[1]{\textcolor[rgb]{0.00,0.00,0.81}{#1}}
\newcommand{\DocumentationTok}[1]{\textcolor[rgb]{0.56,0.35,0.01}{\textbf{\textit{#1}}}}
\newcommand{\ErrorTok}[1]{\textcolor[rgb]{0.64,0.00,0.00}{\textbf{#1}}}
\newcommand{\ExtensionTok}[1]{#1}
\newcommand{\FloatTok}[1]{\textcolor[rgb]{0.00,0.00,0.81}{#1}}
\newcommand{\FunctionTok}[1]{\textcolor[rgb]{0.00,0.00,0.00}{#1}}
\newcommand{\ImportTok}[1]{#1}
\newcommand{\InformationTok}[1]{\textcolor[rgb]{0.56,0.35,0.01}{\textbf{\textit{#1}}}}
\newcommand{\KeywordTok}[1]{\textcolor[rgb]{0.13,0.29,0.53}{\textbf{#1}}}
\newcommand{\NormalTok}[1]{#1}
\newcommand{\OperatorTok}[1]{\textcolor[rgb]{0.81,0.36,0.00}{\textbf{#1}}}
\newcommand{\OtherTok}[1]{\textcolor[rgb]{0.56,0.35,0.01}{#1}}
\newcommand{\PreprocessorTok}[1]{\textcolor[rgb]{0.56,0.35,0.01}{\textit{#1}}}
\newcommand{\RegionMarkerTok}[1]{#1}
\newcommand{\SpecialCharTok}[1]{\textcolor[rgb]{0.00,0.00,0.00}{#1}}
\newcommand{\SpecialStringTok}[1]{\textcolor[rgb]{0.31,0.60,0.02}{#1}}
\newcommand{\StringTok}[1]{\textcolor[rgb]{0.31,0.60,0.02}{#1}}
\newcommand{\VariableTok}[1]{\textcolor[rgb]{0.00,0.00,0.00}{#1}}
\newcommand{\VerbatimStringTok}[1]{\textcolor[rgb]{0.31,0.60,0.02}{#1}}
\newcommand{\WarningTok}[1]{\textcolor[rgb]{0.56,0.35,0.01}{\textbf{\textit{#1}}}}
\usepackage{graphicx}
\makeatletter
\def\maxwidth{\ifdim\Gin@nat@width>\linewidth\linewidth\else\Gin@nat@width\fi}
\def\maxheight{\ifdim\Gin@nat@height>\textheight\textheight\else\Gin@nat@height\fi}
\makeatother
% Scale images if necessary, so that they will not overflow the page
% margins by default, and it is still possible to overwrite the defaults
% using explicit options in \includegraphics[width, height, ...]{}
\setkeys{Gin}{width=\maxwidth,height=\maxheight,keepaspectratio}
% Set default figure placement to htbp
\makeatletter
\def\fps@figure{htbp}
\makeatother
\setlength{\emergencystretch}{3em} % prevent overfull lines
\providecommand{\tightlist}{%
  \setlength{\itemsep}{0pt}\setlength{\parskip}{0pt}}
\setcounter{secnumdepth}{-\maxdimen} % remove section numbering
\ifLuaTeX
  \usepackage{selnolig}  % disable illegal ligatures
\fi

\begin{document}
\maketitle

\begin{Shaded}
\begin{Highlighting}[]
\DocumentationTok{\#\#\#\#\#\#\#\#\#\#\#\#\#\#\#\#\#\#\#\#\#\#\#\#\#\#\#\#\#\#\#\#\#\#\#\# SCRIPT \#\#\#\#\#\#\#\#\#\#\#\#\#\#\#\#\#\#\#\#\#\#\#\#\#\#\#\#\#\#\#\#\#\#\#\#}

\CommentTok{\# Borrar objetos de la memoria}
\FunctionTok{rm}\NormalTok{(}\AttributeTok{list =} \FunctionTok{ls}\NormalTok{()) }

\DocumentationTok{\#\#\#\#\#\#\#\#\#\#\#\#\#\#\#\#\#\#\#\#\#\#\#\#\#\#\#\#\#\#\#\#\#\#\#\#\#\#\#\#\#\#\#\#\#\#\#\#\#\#\#\#\#\#\#\#\#\#\#\#\#\#\#\#\#\#\#\#\#\#\#\#\#\#\#\#\#\#\#\#}
\DocumentationTok{\#\#\#\#\#\#\#\#\#\#\#\#\#\#\#\#\#\#\#\#\#\#\# CARGA DE LIBRERIAS Y DATOS: \#\#\#\#\#\#\#\#\#\#\#\#\#\#\#\#\#\#\#\#\#\#\#\#\#\#\#\#}
\DocumentationTok{\#\#\#\#\#\#\#\#\#\#\#\#\#\#\#\#\#\#\#\#\#\#\#\#\#\#\#\#\#\#\#\#\#\#\#\#\#\#\#\#\#\#\#\#\#\#\#\#\#\#\#\#\#\#\#\#\#\#\#\#\#\#\#\#\#\#\#\#\#\#\#\#\#\#\#\#\#\#\#\#}

\FunctionTok{library}\NormalTok{(readxl)    }\CommentTok{\# datos de excel}
\FunctionTok{library}\NormalTok{(reshape2)  }\CommentTok{\# melt}
\FunctionTok{library}\NormalTok{(pastecs)   }\CommentTok{\# tapply}
\FunctionTok{library}\NormalTok{(ggplot2)   }\CommentTok{\# gráficos}
\FunctionTok{library}\NormalTok{(ggeffects) }\CommentTok{\# ggpredict (todavía no lo usamos)}
\FunctionTok{library}\NormalTok{(dplyr)}
\end{Highlighting}
\end{Shaded}

\begin{verbatim}
## 
## Attaching package: 'dplyr'
\end{verbatim}

\begin{verbatim}
## The following objects are masked from 'package:pastecs':
## 
##     first, last
\end{verbatim}

\begin{verbatim}
## The following objects are masked from 'package:stats':
## 
##     filter, lag
\end{verbatim}

\begin{verbatim}
## The following objects are masked from 'package:base':
## 
##     intersect, setdiff, setequal, union
\end{verbatim}

\begin{Shaded}
\begin{Highlighting}[]
\FunctionTok{library}\NormalTok{(geepack)   }\CommentTok{\# modelado con geeglm}
\FunctionTok{library}\NormalTok{(MuMIn)     }\CommentTok{\# model.sel}
\end{Highlighting}
\end{Shaded}

\begin{verbatim}
## 
## Attaching package: 'MuMIn'
\end{verbatim}

\begin{verbatim}
## The following object is masked from 'package:geepack':
## 
##     QIC
\end{verbatim}

\begin{Shaded}
\begin{Highlighting}[]
\FunctionTok{library}\NormalTok{(glmmTMB)   }\CommentTok{\# modelado con glmmTMB}
\end{Highlighting}
\end{Shaded}

\begin{verbatim}
## Warning in checkMatrixPackageVersion(): Package version inconsistency detected.
## TMB was built with Matrix version 1.3.3
## Current Matrix version is 1.2.18
## Please re-install 'TMB' from source using install.packages('TMB', type = 'source') or ask CRAN for a binary version of 'TMB' matching CRAN's 'Matrix' package
\end{verbatim}

\begin{Shaded}
\begin{Highlighting}[]
\FunctionTok{library}\NormalTok{(car)       }\CommentTok{\# Anova}
\end{Highlighting}
\end{Shaded}

\begin{verbatim}
## Loading required package: carData
\end{verbatim}

\begin{verbatim}
## 
## Attaching package: 'car'
\end{verbatim}

\begin{verbatim}
## The following object is masked from 'package:dplyr':
## 
##     recode
\end{verbatim}

\begin{Shaded}
\begin{Highlighting}[]
\FunctionTok{library}\NormalTok{(emmeans)   }\CommentTok{\# comparaciones}
\FunctionTok{library}\NormalTok{(DHARMa)}
\end{Highlighting}
\end{Shaded}

\begin{verbatim}
## This is DHARMa 0.4.4. For overview type '?DHARMa'. For recent changes, type news(package = 'DHARMa')
\end{verbatim}

\begin{Shaded}
\begin{Highlighting}[]
\FunctionTok{setwd}\NormalTok{(}\StringTok{"C:/Users/hecto/Desktop/Ilán/Biome II/TP FINAL/TP{-}FINAL{-}BIOME"}\NormalTok{)}
\NormalTok{datos }\OtherTok{\textless{}{-}} \FunctionTok{read\_excel}\NormalTok{(}\StringTok{"datos.xlsx"}\NormalTok{)}
\FunctionTok{str}\NormalTok{(datos)}
\end{Highlighting}
\end{Shaded}

\begin{verbatim}
## tibble [132 x 12] (S3: tbl_df/tbl/data.frame)
##  $ SEMANA     : num [1:132] 1 1 1 1 1 1 1 1 1 1 ...
##  $ ID         : num [1:132] 1 2 3 4 5 6 7 8 9 10 ...
##  $ ANT        : chr [1:132] "A_0.5" "A_0.5" "A_0.5" "A_0.5" ...
##  $ PROB       : chr [1:132] "P_0.5" "P_0.5" "P_0.5" "P_0.5" ...
##  $ TRATAMIENTO: chr [1:132] "constante_bajo" "constante_bajo" "constante_bajo" "constante_bajo" ...
##  $ E1         : num [1:132] 0 0 0 0 0 0 0 0 0 0 ...
##  $ E2         : num [1:132] 1 1 1 1 0 0 1 1 1 1 ...
##  $ E3         : num [1:132] 1 1 1 1 1 0 1 1 1 1 ...
##  $ E4         : num [1:132] 1 1 1 1 1 0 1 0 1 1 ...
##  $ 3hs        : num [1:132] 1 1 1 1 1 0 1 1 1 1 ...
##  $ 24hs       : num [1:132] 1 0 0 0 0 0 1 0 1 1 ...
##  $ 48hs       : num [1:132] 1 0 0 0 1 0 1 1 0 1 ...
\end{verbatim}

\begin{Shaded}
\begin{Highlighting}[]
\NormalTok{datos}\SpecialCharTok{$}\NormalTok{SEMANA         }\OtherTok{\textless{}{-}} \FunctionTok{as.factor}\NormalTok{(datos}\SpecialCharTok{$}\NormalTok{SEMANA)}
\NormalTok{datos}\SpecialCharTok{$}\NormalTok{ID             }\OtherTok{\textless{}{-}} \FunctionTok{as.factor}\NormalTok{(datos}\SpecialCharTok{$}\NormalTok{ID)}
\NormalTok{datos}\SpecialCharTok{$}\NormalTok{ANT            }\OtherTok{\textless{}{-}} \FunctionTok{as.factor}\NormalTok{(datos}\SpecialCharTok{$}\NormalTok{ANT)}
\NormalTok{datos}\SpecialCharTok{$}\NormalTok{PROB           }\OtherTok{\textless{}{-}} \FunctionTok{as.factor}\NormalTok{(datos}\SpecialCharTok{$}\NormalTok{PROB)}
\NormalTok{datos}\SpecialCharTok{$}\NormalTok{TRATAMIENTO    }\OtherTok{\textless{}{-}} \FunctionTok{as.factor}\NormalTok{(datos}\SpecialCharTok{$}\NormalTok{TRATAMIENTO)}
\FunctionTok{str}\NormalTok{(datos)}
\end{Highlighting}
\end{Shaded}

\begin{verbatim}
## tibble [132 x 12] (S3: tbl_df/tbl/data.frame)
##  $ SEMANA     : Factor w/ 7 levels "1","2","3","4",..: 1 1 1 1 1 1 1 1 1 1 ...
##  $ ID         : Factor w/ 132 levels "1","2","3","4",..: 1 2 3 4 5 6 7 8 9 10 ...
##  $ ANT        : Factor w/ 2 levels "A_0.5","A_1.5": 1 1 1 1 2 2 2 2 2 1 ...
##  $ PROB       : Factor w/ 2 levels "P_0.5","P_1.5": 1 1 1 1 2 2 2 2 2 2 ...
##  $ TRATAMIENTO: Factor w/ 4 levels "constante_alto",..: 2 2 2 2 1 1 1 1 1 4 ...
##  $ E1         : num [1:132] 0 0 0 0 0 0 0 0 0 0 ...
##  $ E2         : num [1:132] 1 1 1 1 0 0 1 1 1 1 ...
##  $ E3         : num [1:132] 1 1 1 1 1 0 1 1 1 1 ...
##  $ E4         : num [1:132] 1 1 1 1 1 0 1 0 1 1 ...
##  $ 3hs        : num [1:132] 1 1 1 1 1 0 1 1 1 1 ...
##  $ 24hs       : num [1:132] 1 0 0 0 0 0 1 0 1 1 ...
##  $ 48hs       : num [1:132] 1 0 0 0 1 0 1 1 0 1 ...
\end{verbatim}

\begin{Shaded}
\begin{Highlighting}[]
\CommentTok{\# Colmena no la vamos a incluir en el analisis porque esta explicada por Semana.}

\DocumentationTok{\#\#\#\#\#\#\#\#\#\#\#\#\#\#\#\#\#\#\#\#\#\#\#\#\#\#\#\#\#\#\#\#\#\#\#\#\#\#\#\#\#\#\#\#\#\#\#\#\#\#\#\#\#\#\#\#\#\#\#\#\#\#\#\#\#\#\#\#\#\#\#\#\#\#\#\#\#\#\#\#}
\DocumentationTok{\#\#\#\#\#\#\#\#\#\#\#\#\#\#\#\#\#\#\#\#\#\#\#\#\#\#\#\#\#\#\# DESCRIPTIVA \#\#\#\#\#\#\#\#\#\#\#\#\#\#\#\#\#\#\#\#\#\#\#\#\#\#\#\#\#\#\#\#\#\#\#\#}
\DocumentationTok{\#\#\#\#\#\#\#\#\#\#\#\#\#\#\#\#\#\#\#\#\#\#\#\#\#\#\#\#\#\#\#\#\#\#\#\#\#\#\#\#\#\#\#\#\#\#\#\#\#\#\#\#\#\#\#\#\#\#\#\#\#\#\#\#\#\#\#\#\#\#\#\#\#\#\#\#\#\#\#\#}

\DocumentationTok{\#\#\#\#\#\#\#\#\#\#\#\#\#\#\#\#\#\# ENTRENAMIENTO:}
\CommentTok{\# PASAMOS DATOS ENTRENAMIENTO A LONG:}
\NormalTok{wide\_entr }\OtherTok{\textless{}{-}}\NormalTok{ datos[,}\FunctionTok{c}\NormalTok{(}\DecValTok{1}\NormalTok{,}\DecValTok{2}\NormalTok{,}\DecValTok{3}\NormalTok{,}\DecValTok{4}\NormalTok{,}\DecValTok{5}\NormalTok{,}\DecValTok{6}\NormalTok{,}\DecValTok{7}\NormalTok{,}\DecValTok{8}\NormalTok{,}\DecValTok{9}\NormalTok{)]}
\NormalTok{long\_entr }\OtherTok{\textless{}{-}} \FunctionTok{melt}\NormalTok{(wide\_entr,}
                \AttributeTok{id.vars =} \FunctionTok{c}\NormalTok{(}\StringTok{"SEMANA"}\NormalTok{, }\StringTok{"ID"}\NormalTok{, }\StringTok{"ANT"}\NormalTok{, }\StringTok{"PROB"}\NormalTok{, }\StringTok{"TRATAMIENTO"}\NormalTok{),}
                \AttributeTok{variable.name =} \StringTok{"tiempo\_entr"}\NormalTok{,}
                \AttributeTok{value.name =} \StringTok{"rta"}\NormalTok{)}

\CommentTok{\# PROBABILIDAD DE EXITO SEGUN TRATAMIENTO Y TIEMPO:}
\NormalTok{prob\_exito\_entr }\OtherTok{\textless{}{-}} \FunctionTok{round}\NormalTok{(}\FunctionTok{tapply}\NormalTok{(long\_entr}\SpecialCharTok{$}\NormalTok{rta,}\FunctionTok{list}\NormalTok{(long\_entr}\SpecialCharTok{$}\NormalTok{TRATAMIENTO,long\_entr}\SpecialCharTok{$}\NormalTok{tiempo\_entr),mean),}\DecValTok{2}\NormalTok{)}
\NormalTok{prob\_exito\_entr}
\end{Highlighting}
\end{Shaded}

\begin{verbatim}
##                E1   E2   E3   E4
## constante_alto  0 0.48 0.65 0.48
## constante_bajo  0 0.61 0.61 0.55
## contraste_neg   0 0.48 0.52 0.67
## contraste_pos   0 0.62 0.68 0.57
\end{verbatim}

\begin{Shaded}
\begin{Highlighting}[]
\CommentTok{\# Creamos un data frame en formato long con estos valores:}
\NormalTok{prob\_exito\_entr\_long }\OtherTok{\textless{}{-}} \FunctionTok{as.data.frame.table}\NormalTok{(prob\_exito\_entr)}
\FunctionTok{colnames}\NormalTok{(prob\_exito\_entr\_long) }\OtherTok{\textless{}{-}} \FunctionTok{c}\NormalTok{(}\StringTok{"TRATAMIENTO"}\NormalTok{,}\StringTok{"nro\_ensayo"}\NormalTok{,}\StringTok{"proporcion\_exitos"}\NormalTok{)}

\CommentTok{\# Reordenamos los levels para la leyenda del grafico:}
\NormalTok{prob\_exito\_entr\_long}\SpecialCharTok{$}\NormalTok{TRATAMIENTO }\OtherTok{\textless{}{-}} \FunctionTok{factor}\NormalTok{(prob\_exito\_entr\_long}\SpecialCharTok{$}\NormalTok{TRATAMIENTO, }\AttributeTok{levels =} \FunctionTok{c}\NormalTok{(}\StringTok{"contraste\_pos"}\NormalTok{,}\StringTok{"constante\_alto"}\NormalTok{, }\StringTok{"constante\_bajo"}\NormalTok{,}\StringTok{"contraste\_neg"}\NormalTok{))}

\CommentTok{\# Graficamente:}
\NormalTok{gp\_entr }\OtherTok{\textless{}{-}} \FunctionTok{ggplot}\NormalTok{(prob\_exito\_entr\_long, }\FunctionTok{aes}\NormalTok{(}\AttributeTok{x=}\NormalTok{nro\_ensayo, }\AttributeTok{y=}\NormalTok{proporcion\_exitos, }\AttributeTok{colour=}\NormalTok{TRATAMIENTO, }\AttributeTok{group=}\NormalTok{TRATAMIENTO)) }\SpecialCharTok{+}
  \FunctionTok{geom\_line}\NormalTok{(}\AttributeTok{linetype=}\DecValTok{2}\NormalTok{) }\SpecialCharTok{+}
  \FunctionTok{geom\_point}\NormalTok{(}\AttributeTok{size=}\DecValTok{2}\NormalTok{,}\AttributeTok{shape=}\FunctionTok{c}\NormalTok{(}\DecValTok{15}\NormalTok{,}\DecValTok{16}\NormalTok{,}\DecValTok{16}\NormalTok{,}\DecValTok{15}\NormalTok{,}\DecValTok{15}\NormalTok{,}\DecValTok{16}\NormalTok{,}\DecValTok{16}\NormalTok{,}\DecValTok{15}\NormalTok{,}\DecValTok{15}\NormalTok{,}\DecValTok{16}\NormalTok{,}\DecValTok{16}\NormalTok{,}\DecValTok{15}\NormalTok{,}\DecValTok{15}\NormalTok{,}\DecValTok{16}\NormalTok{,}\DecValTok{16}\NormalTok{,}\DecValTok{15}\NormalTok{)) }\SpecialCharTok{+} 
  \CommentTok{\#shape puntos}
  \FunctionTok{labs}\NormalTok{(}\AttributeTok{x=}\StringTok{"Numero de ensayo"}\NormalTok{,}\AttributeTok{y=}\StringTok{"Proporcion de PER"}\NormalTok{,}\AttributeTok{title=}\StringTok{"Curva de aprendizaje"}\NormalTok{) }\SpecialCharTok{+}
  \FunctionTok{ylim}\NormalTok{(}\DecValTok{0}\NormalTok{,}\DecValTok{1}\NormalTok{) }\SpecialCharTok{+} \FunctionTok{theme\_classic}\NormalTok{() }\SpecialCharTok{+}
  \FunctionTok{scale\_colour\_manual}\NormalTok{(}\AttributeTok{labels =} \FunctionTok{c}\NormalTok{(}\StringTok{"Contraste positivo"}\NormalTok{,}\StringTok{"Constante alto"}\NormalTok{,}
                                 \StringTok{"Constante bajo"}\NormalTok{,}\StringTok{"Contraste negativo"}\NormalTok{),}
                      \AttributeTok{values=}\FunctionTok{c}\NormalTok{(}\StringTok{"\#00b050"}\NormalTok{,}\StringTok{"\#70ad47"}\NormalTok{,}\StringTok{"\#ed7c31"}\NormalTok{,}\StringTok{"\#ff0000"}\NormalTok{)) }\SpecialCharTok{+}
  \FunctionTok{labs}\NormalTok{(}\AttributeTok{col=}\StringTok{"Tratamiento"}\NormalTok{) }\SpecialCharTok{+}
  \FunctionTok{guides}\NormalTok{(}\AttributeTok{color =} \FunctionTok{guide\_legend}\NormalTok{(}\AttributeTok{override.aes=}\FunctionTok{list}\NormalTok{(}\AttributeTok{shape=}\FunctionTok{c}\NormalTok{(}\DecValTok{15}\NormalTok{,}\DecValTok{15}\NormalTok{,}\DecValTok{16}\NormalTok{,}\DecValTok{16}\NormalTok{)))) }\CommentTok{\#leyenda}
\NormalTok{gp\_entr}
\end{Highlighting}
\end{Shaded}

\includegraphics{TP_FINAL_RMarkdown_files/figure-latex/unnamed-chunk-1-1.pdf}

\begin{Shaded}
\begin{Highlighting}[]
\DocumentationTok{\#\#\#\#\#\#\#\#\#\#\#\#\#\#\#\#\#\# TESTEO:}
\CommentTok{\# PASAMOS DATOS }\AlertTok{TEST}\CommentTok{ A LONG:}
\NormalTok{wide\_testeo }\OtherTok{\textless{}{-}}\NormalTok{ datos[,}\FunctionTok{c}\NormalTok{(}\DecValTok{1}\NormalTok{,}\DecValTok{2}\NormalTok{,}\DecValTok{3}\NormalTok{,}\DecValTok{4}\NormalTok{,}\DecValTok{5}\NormalTok{,}\DecValTok{10}\NormalTok{,}\DecValTok{11}\NormalTok{,}\DecValTok{12}\NormalTok{)]}
\NormalTok{long\_testeo }\OtherTok{\textless{}{-}} \FunctionTok{melt}\NormalTok{(wide\_testeo,}
                  \AttributeTok{id.vars =} \FunctionTok{c}\NormalTok{(}\StringTok{"SEMANA"}\NormalTok{, }\StringTok{"ID"}\NormalTok{, }\StringTok{"ANT"}\NormalTok{, }\StringTok{"PROB"}\NormalTok{, }\StringTok{"TRATAMIENTO"}\NormalTok{),}
                  \AttributeTok{variable.name =} \StringTok{"tiempo\_testeo"}\NormalTok{,}
                  \AttributeTok{value.name =} \StringTok{"rta"}\NormalTok{)}

\CommentTok{\# PROBABILIDAD DE EXITO SEGUN TRATAMIENTO Y TIEMPO EN TESTEO:}
\NormalTok{prob\_exito\_testeo }\OtherTok{\textless{}{-}} \FunctionTok{round}\NormalTok{(}\FunctionTok{tapply}\NormalTok{(long\_testeo}\SpecialCharTok{$}\NormalTok{rta,}\FunctionTok{list}\NormalTok{(long\_testeo}\SpecialCharTok{$}\NormalTok{TRATAMIENTO,long\_testeo}\SpecialCharTok{$}\NormalTok{tiempo\_testeo), mean),}\DecValTok{2}\NormalTok{)}
\NormalTok{prob\_exito\_testeo}
\end{Highlighting}
\end{Shaded}

\begin{verbatim}
##                 3hs 24hs 48hs
## constante_alto 0.45 0.45 0.52
## constante_bajo 0.61 0.48 0.39
## contraste_neg  0.70 0.42 0.18
## contraste_pos  0.51 0.70 0.70
\end{verbatim}

\begin{Shaded}
\begin{Highlighting}[]
\CommentTok{\# Creamos un data frame en formato long con estos valores:}
\NormalTok{prob\_exito\_testeo\_long }\OtherTok{\textless{}{-}} \FunctionTok{as.data.frame.table}\NormalTok{(prob\_exito\_testeo)}
\FunctionTok{colnames}\NormalTok{(prob\_exito\_testeo\_long) }\OtherTok{\textless{}{-}} \FunctionTok{c}\NormalTok{(}\StringTok{"TRATAMIENTO"}\NormalTok{,}\StringTok{"tiempo\_testeo"}\NormalTok{,}\StringTok{"proporcion\_exitos"}\NormalTok{)}

\CommentTok{\# Cambiamos los nombres del tiempo de testeo:}
\NormalTok{prob\_exito\_testeo\_long}\SpecialCharTok{$}\NormalTok{tiempo\_testeo }\OtherTok{\textless{}{-}} \FunctionTok{with}\NormalTok{(prob\_exito\_testeo\_long, }\FunctionTok{factor}\NormalTok{(tiempo\_testeo,}\AttributeTok{levels =} \FunctionTok{c}\NormalTok{(}\StringTok{"3hs"}\NormalTok{,}\StringTok{"24hs"}\NormalTok{,}\StringTok{"48hs"}\NormalTok{),}\AttributeTok{labels =} \FunctionTok{c}\NormalTok{(}\StringTok{"3"}\NormalTok{,}\StringTok{"24"}\NormalTok{,}\StringTok{"48"}\NormalTok{)))}

\CommentTok{\# Reordenamos los levels para la leyenda del grafico:}
\NormalTok{prob\_exito\_testeo\_long}\SpecialCharTok{$}\NormalTok{TRATAMIENTO }\OtherTok{\textless{}{-}} \FunctionTok{factor}\NormalTok{(prob\_exito\_testeo\_long}\SpecialCharTok{$}\NormalTok{TRATAMIENTO, }\AttributeTok{levels =} \FunctionTok{c}\NormalTok{(}\StringTok{"contraste\_pos"}\NormalTok{,}\StringTok{"constante\_alto"}\NormalTok{, }\StringTok{"constante\_bajo"}\NormalTok{,}\StringTok{"contraste\_neg"}\NormalTok{))}

\CommentTok{\# Graficamente:}
\NormalTok{gp\_testeo }\OtherTok{\textless{}{-}} \FunctionTok{ggplot}\NormalTok{(prob\_exito\_testeo\_long,}\FunctionTok{aes}\NormalTok{(}\AttributeTok{x=}\NormalTok{tiempo\_testeo, }\AttributeTok{y=}\NormalTok{proporcion\_exitos, }\AttributeTok{colour=}\NormalTok{TRATAMIENTO,}\AttributeTok{group=}\NormalTok{TRATAMIENTO)) }\SpecialCharTok{+}
  \FunctionTok{geom\_line}\NormalTok{(}\AttributeTok{linetype=}\DecValTok{2}\NormalTok{) }\SpecialCharTok{+}
  \FunctionTok{geom\_point}\NormalTok{(}\AttributeTok{size=}\DecValTok{2}\NormalTok{,}\AttributeTok{shape=}\FunctionTok{c}\NormalTok{(}\DecValTok{15}\NormalTok{,}\DecValTok{16}\NormalTok{,}\DecValTok{16}\NormalTok{,}\DecValTok{15}\NormalTok{,}\DecValTok{15}\NormalTok{,}\DecValTok{16}\NormalTok{,}\DecValTok{16}\NormalTok{,}\DecValTok{15}\NormalTok{,}\DecValTok{15}\NormalTok{,}\DecValTok{16}\NormalTok{,}\DecValTok{16}\NormalTok{,}\DecValTok{15}\NormalTok{)) }\SpecialCharTok{+} \CommentTok{\#shape puntos}
  \FunctionTok{labs}\NormalTok{(}\AttributeTok{x=}\StringTok{"Tiempo de evaluacion (hs)"}\NormalTok{,}\AttributeTok{y=}\StringTok{"Proporcion de PER"}\NormalTok{,}
       \AttributeTok{title=}\StringTok{"Descriptiva evaluacion"}\NormalTok{) }\SpecialCharTok{+}
  \FunctionTok{ylim}\NormalTok{(}\DecValTok{0}\NormalTok{,}\DecValTok{1}\NormalTok{) }\SpecialCharTok{+} \FunctionTok{theme\_classic}\NormalTok{() }\SpecialCharTok{+}
  \FunctionTok{scale\_colour\_manual}\NormalTok{(}\AttributeTok{labels =} \FunctionTok{c}\NormalTok{(}\StringTok{"Contraste positivo"}\NormalTok{,}\StringTok{"Constante alto"}\NormalTok{,}
                                 \StringTok{"Constante bajo"}\NormalTok{,}\StringTok{"Contraste negativo"}\NormalTok{),}
                      \AttributeTok{values=}\FunctionTok{c}\NormalTok{(}\StringTok{"\#00b050"}\NormalTok{,}\StringTok{"\#70ad47"}\NormalTok{,}\StringTok{"\#ed7c31"}\NormalTok{,}\StringTok{"\#ff0000"}\NormalTok{)) }\SpecialCharTok{+}
  \FunctionTok{guides}\NormalTok{(}\AttributeTok{color =} \FunctionTok{guide\_legend}\NormalTok{(}\AttributeTok{override.aes=}\FunctionTok{list}\NormalTok{(}\AttributeTok{shape=}\FunctionTok{c}\NormalTok{(}\DecValTok{15}\NormalTok{,}\DecValTok{15}\NormalTok{,}\DecValTok{16}\NormalTok{,}\DecValTok{16}\NormalTok{)))) }\SpecialCharTok{+} \CommentTok{\#leyenda}
  \FunctionTok{labs}\NormalTok{(}\AttributeTok{col=}\StringTok{"Tratamiento"}\NormalTok{)}
\NormalTok{gp\_testeo}
\end{Highlighting}
\end{Shaded}

\includegraphics{TP_FINAL_RMarkdown_files/figure-latex/unnamed-chunk-1-2.pdf}

\begin{Shaded}
\begin{Highlighting}[]
\DocumentationTok{\#\#\#\#\#\#\#\#\#\#\#\#\#\#\#\#\#\# EXPLORAMOS SEMANA COMO COVARIABLE:}

\NormalTok{prob\_exito\_semana }\OtherTok{\textless{}{-}} \FunctionTok{round}\NormalTok{(}\FunctionTok{tapply}\NormalTok{(long\_testeo}\SpecialCharTok{$}\NormalTok{rta,}\FunctionTok{list}\NormalTok{(long\_testeo}\SpecialCharTok{$}\NormalTok{SEMANA,long\_testeo}\SpecialCharTok{$}\NormalTok{tiempo\_testeo), mean),}\DecValTok{2}\NormalTok{)}
\NormalTok{prob\_exito\_semana}
\end{Highlighting}
\end{Shaded}

\begin{verbatim}
##    3hs 24hs 48hs
## 1 0.90 0.52 0.57
## 2 0.62 0.62 0.69
## 3 0.69 0.62 0.46
## 4 0.79 0.53 0.74
## 5 0.32 0.45 0.41
## 6 0.57 0.64 0.32
## 7 0.06 0.25 0.06
\end{verbatim}

\begin{Shaded}
\begin{Highlighting}[]
\CommentTok{\# Creamos un data frame en formato long con estos valores:}
\NormalTok{prob\_exito\_semana\_long }\OtherTok{\textless{}{-}} \FunctionTok{as.data.frame.table}\NormalTok{(prob\_exito\_semana)}
\FunctionTok{colnames}\NormalTok{(prob\_exito\_semana\_long) }\OtherTok{\textless{}{-}} \FunctionTok{c}\NormalTok{(}\StringTok{"SEMANA"}\NormalTok{,}\StringTok{"tiempo\_testeo"}\NormalTok{,}\StringTok{"proporcion\_exitos"}\NormalTok{)}
\CommentTok{\# Cambiamos los nombres del tiempo de testeo:}
\NormalTok{prob\_exito\_semana\_long}\SpecialCharTok{$}\NormalTok{tiempo\_testeo }\OtherTok{\textless{}{-}} \FunctionTok{with}\NormalTok{(prob\_exito\_semana\_long, }\FunctionTok{factor}\NormalTok{(tiempo\_testeo,}\AttributeTok{levels =} \FunctionTok{c}\NormalTok{(}\StringTok{"3hs"}\NormalTok{,}\StringTok{"24hs"}\NormalTok{,}\StringTok{"48hs"}\NormalTok{),}\AttributeTok{labels =} \FunctionTok{c}\NormalTok{(}\StringTok{"3"}\NormalTok{,}\StringTok{"24"}\NormalTok{,}\StringTok{"48"}\NormalTok{)))}

\CommentTok{\# Graficamente:}
\NormalTok{gp\_semana }\OtherTok{\textless{}{-}} \FunctionTok{ggplot}\NormalTok{(prob\_exito\_semana\_long, }\FunctionTok{aes}\NormalTok{(}\AttributeTok{x=}\NormalTok{tiempo\_testeo, }\AttributeTok{y=}\NormalTok{proporcion\_exitos, }\AttributeTok{colour=}\NormalTok{SEMANA,}\AttributeTok{group=}\NormalTok{SEMANA)) }\SpecialCharTok{+}
  \FunctionTok{geom\_line}\NormalTok{(}\AttributeTok{linetype=}\DecValTok{2}\NormalTok{) }\SpecialCharTok{+}
  \FunctionTok{geom\_point}\NormalTok{(}\AttributeTok{size=}\DecValTok{2}\NormalTok{) }\SpecialCharTok{+}
  \FunctionTok{labs}\NormalTok{(}\AttributeTok{x=}\StringTok{"Tiempo de evaluacion (hs)"}\NormalTok{,}\AttributeTok{y=}\StringTok{"Proporcion de PER"}\NormalTok{,}\AttributeTok{title=}\StringTok{"Proporcion de PER en cada semana"}\NormalTok{,}\AttributeTok{col=}\StringTok{"Semana"}\NormalTok{) }\SpecialCharTok{+}
  \FunctionTok{ylim}\NormalTok{(}\DecValTok{0}\NormalTok{,}\DecValTok{1}\NormalTok{) }\SpecialCharTok{+} \FunctionTok{theme\_classic}\NormalTok{()}
\NormalTok{gp\_semana}
\end{Highlighting}
\end{Shaded}

\includegraphics{TP_FINAL_RMarkdown_files/figure-latex/unnamed-chunk-1-3.pdf}

\begin{Shaded}
\begin{Highlighting}[]
\DocumentationTok{\#\#\#\#\#\#\#\#\#\#\#\#\#\#\#\#\#\#\#\#\#\#\#\#\#\#\#\#\#\#\#\#\#\#\#\#\#\#\#\#\#\#\#\#\#\#\#\#\#\#\#\#\#\#\#\#\#\#\#\#\#\#\#\#\#\#\#\#\#\#\#\#\#\#\#\#\#\#\#\#}
\DocumentationTok{\#\#\#\#\#\#\#\#\#\#\#\#\#\#\#\#\#\#\#\#\#\#\#\#\#\#\#\#\#\#\# MODELADO \#\#\#\#\#\#\#\#\#\#\#\#\#\#\#\#\#\#\#\#\#\#\#\#\#\#\#\#\#\#\#\#\#\#\#\#\#\#\#}
\DocumentationTok{\#\#\#\#\#\#\#\#\#\#\#\#\#\#\#\#\#\#\#\#\#\#\#\#\#\#\#\#\#\#\#\#\#\#\#\#\#\#\#\#\#\#\#\#\#\#\#\#\#\#\#\#\#\#\#\#\#\#\#\#\#\#\#\#\#\#\#\#\#\#\#\#\#\#\#\#\#\#\#\#}

\DocumentationTok{\#\#\#\#\#\#\#\#\#\#\#\#\#\#\#\#\#\# PRIMERA PROPUESTA: MODELO MARGINAL (GEEGLM)}
\CommentTok{\# Para geeglm, las filas del data frame tienen que estar ordenadas por paciente y por tiempo:}
\NormalTok{long\_testeo }\OtherTok{\textless{}{-}} \FunctionTok{arrange}\NormalTok{(long\_testeo,ID)}

\CommentTok{\# Notacion de matrices de covarianza en geeglm:}
\CommentTok{\# Estructura simple: independence}
\CommentTok{\# Simetria compuesta: exchangeable}
\CommentTok{\# AR1: ar1}
\CommentTok{\# Desestructurada: unstructured}

\DocumentationTok{\#\#\# Interaccion tratamiento*tiempo}
\NormalTok{m5 }\OtherTok{\textless{}{-}} \FunctionTok{geeglm}\NormalTok{(}\AttributeTok{formula=}\NormalTok{rta}\SpecialCharTok{\textasciitilde{}}\NormalTok{TRATAMIENTO}\SpecialCharTok{*}\NormalTok{tiempo\_testeo}\SpecialCharTok{+}\NormalTok{SEMANA,}\AttributeTok{family=}\NormalTok{binomial,}\AttributeTok{data=}\NormalTok{long\_testeo,}\AttributeTok{id=}\NormalTok{ID,}
             \AttributeTok{corstr=}\StringTok{"independence"}\NormalTok{)}
\FunctionTok{anova}\NormalTok{(m5)}
\end{Highlighting}
\end{Shaded}

\begin{verbatim}
## Analysis of 'Wald statistic' Table
## Model: binomial, link: logit
## Response: rta
## Terms added sequentially (first to last)
## 
##                           Df     X2 P(>|Chi|)    
## TRATAMIENTO                3  6.051 0.1091733    
## tiempo_testeo              2  5.027 0.0809670 .  
## SEMANA                     6 24.601 0.0004046 ***
## TRATAMIENTO:tiempo_testeo  6 36.992 1.767e-06 ***
## ---
## Signif. codes:  0 '***' 0.001 '**' 0.01 '*' 0.05 '.' 0.1 ' ' 1
\end{verbatim}

\begin{Shaded}
\begin{Highlighting}[]
\NormalTok{m6 }\OtherTok{\textless{}{-}} \FunctionTok{geeglm}\NormalTok{(}\AttributeTok{formula=}\NormalTok{rta}\SpecialCharTok{\textasciitilde{}}\NormalTok{TRATAMIENTO}\SpecialCharTok{*}\NormalTok{tiempo\_testeo}\SpecialCharTok{+}\NormalTok{SEMANA,}\AttributeTok{family=}\NormalTok{binomial,}\AttributeTok{data=}\NormalTok{long\_testeo,}\AttributeTok{id=}\NormalTok{ID,}
             \AttributeTok{corstr=}\StringTok{"exchangeable"}\NormalTok{)}
\FunctionTok{anova}\NormalTok{(m6)}
\end{Highlighting}
\end{Shaded}

\begin{verbatim}
## Analysis of 'Wald statistic' Table
## Model: binomial, link: logit
## Response: rta
## Terms added sequentially (first to last)
## 
##                           Df     X2 P(>|Chi|)    
## TRATAMIENTO                3  6.051 0.1091733    
## tiempo_testeo              2  5.019 0.0813146 .  
## SEMANA                     6 24.316 0.0004568 ***
## TRATAMIENTO:tiempo_testeo  6 36.155 2.572e-06 ***
## ---
## Signif. codes:  0 '***' 0.001 '**' 0.01 '*' 0.05 '.' 0.1 ' ' 1
\end{verbatim}

\begin{Shaded}
\begin{Highlighting}[]
\NormalTok{m7 }\OtherTok{\textless{}{-}} \FunctionTok{geeglm}\NormalTok{(}\AttributeTok{formula=}\NormalTok{rta}\SpecialCharTok{\textasciitilde{}}\NormalTok{TRATAMIENTO}\SpecialCharTok{*}\NormalTok{tiempo\_testeo}\SpecialCharTok{+}\NormalTok{SEMANA,}\AttributeTok{family=}\NormalTok{binomial,}\AttributeTok{data=}\NormalTok{long\_testeo,}\AttributeTok{id=}\NormalTok{ID,}
             \AttributeTok{corstr=}\StringTok{"ar1"}\NormalTok{)}
\FunctionTok{anova}\NormalTok{(m7)}
\end{Highlighting}
\end{Shaded}

\begin{verbatim}
## Analysis of 'Wald statistic' Table
## Model: binomial, link: logit
## Response: rta
## Terms added sequentially (first to last)
## 
##                           Df     X2 P(>|Chi|)    
## TRATAMIENTO                3  5.337   0.14869    
## tiempo_testeo              2  5.109   0.07772 .  
## SEMANA                     6 28.542 7.426e-05 ***
## TRATAMIENTO:tiempo_testeo  6 35.045 4.225e-06 ***
## ---
## Signif. codes:  0 '***' 0.001 '**' 0.01 '*' 0.05 '.' 0.1 ' ' 1
\end{verbatim}

\begin{Shaded}
\begin{Highlighting}[]
\NormalTok{m8 }\OtherTok{\textless{}{-}} \FunctionTok{geeglm}\NormalTok{(}\AttributeTok{formula=}\NormalTok{rta}\SpecialCharTok{\textasciitilde{}}\NormalTok{TRATAMIENTO}\SpecialCharTok{*}\NormalTok{tiempo\_testeo}\SpecialCharTok{+}\NormalTok{SEMANA,}\AttributeTok{family=}\NormalTok{binomial,}\AttributeTok{data=}\NormalTok{long\_testeo,}\AttributeTok{id=}\NormalTok{ID,}
             \AttributeTok{corstr=}\StringTok{"unstructured"}\NormalTok{)}
\FunctionTok{anova}\NormalTok{(m8)}
\end{Highlighting}
\end{Shaded}

\begin{verbatim}
## Analysis of 'Wald statistic' Table
## Model: binomial, link: logit
## Response: rta
## Terms added sequentially (first to last)
## 
##                           Df     X2 P(>|Chi|)    
## TRATAMIENTO                3  5.936   0.11475    
## tiempo_testeo              2  5.057   0.07976 .  
## SEMANA                     6 29.209 5.554e-05 ***
## TRATAMIENTO:tiempo_testeo  6 35.376 3.644e-06 ***
## ---
## Signif. codes:  0 '***' 0.001 '**' 0.01 '*' 0.05 '.' 0.1 ' ' 1
\end{verbatim}

\begin{Shaded}
\begin{Highlighting}[]
\DocumentationTok{\#\# SELECCION DE MODELOS (en geeglm rankeamos por QIC):}
\FunctionTok{model.sel}\NormalTok{(m5,m6,m7,m8, }\AttributeTok{rank =}\NormalTok{ QIC)}
\end{Highlighting}
\end{Shaded}

\begin{verbatim}
## Model selection table 
##     (Int) SEM tmp_tst TRA tmp_tst:TRA corstr     qLik   QIC delta weight
## m5 0.4608   +       +   +           + indpnd -227.246 497.0  0.00  0.310
## m6 0.4283   +       +   +           + exchng -227.326 497.1  0.10  0.294
## m8 0.4809   +       +   +           + unstrc -227.552 497.7  0.65  0.223
## m7 0.5293   +       +   +           +    ar1 -227.769 498.2  1.17  0.173
## Abbreviations:
## corstr: exchng = 'exchangeable', indpnd = 'independence', 
##         unstrc = 'unstructured'
## Models ranked by QIC(x)
\end{verbatim}

\begin{Shaded}
\begin{Highlighting}[]
\CommentTok{\# Estructura simple: la sacamos porque no estariamos declarando dependencia de datos entre tiempos para una misma abeja.}
\CommentTok{\# Simetria compuesta: decimos que para cada abeja hay una misma correlacion entre tiempos.}
\CommentTok{\# AR1: la sacamos porque a pesar de tener la misma cantidad de parámetros que la de simetria compuesta, tiene un peor ajuste porque nuestros tiempos no son equidistantes como asume AR1, entonces el QIC es mayor.}
\CommentTok{\# Desestructurada: la sacamos porque estima mas parametros, por eso el QIC da un poco mas alto.}

\DocumentationTok{\#\#\#\#\#\#\#\#\#\#\#\#\#\#\#\#\#\# SEGUNDA PROPUESTA: MODELO CONDICIONAL (GLMMTMB)}

\CommentTok{\# Como elegimos simetria compuesta, probamos un modelo condicional:}

\CommentTok{\# Semana como aleatoria:}
\NormalTok{m9 }\OtherTok{\textless{}{-}} \FunctionTok{glmmTMB}\NormalTok{(rta }\SpecialCharTok{\textasciitilde{}}\NormalTok{ TRATAMIENTO}\SpecialCharTok{*}\NormalTok{tiempo\_testeo }\SpecialCharTok{+}\NormalTok{  (}\DecValTok{1}\SpecialCharTok{|}\NormalTok{SEMANA) }\SpecialCharTok{+}\NormalTok{ (}\DecValTok{1}\SpecialCharTok{|}\NormalTok{ID), }\AttributeTok{data=}\NormalTok{long\_testeo, }\AttributeTok{family=}\StringTok{"binomial"}\NormalTok{)}

\CommentTok{\# Semana como fija:}
\NormalTok{m10 }\OtherTok{\textless{}{-}} \FunctionTok{glmmTMB}\NormalTok{(rta }\SpecialCharTok{\textasciitilde{}}\NormalTok{ TRATAMIENTO}\SpecialCharTok{*}\NormalTok{tiempo\_testeo }\SpecialCharTok{+}\NormalTok{ SEMANA }\SpecialCharTok{+}\NormalTok{ (}\DecValTok{1}\SpecialCharTok{|}\NormalTok{ID), }\AttributeTok{data=}\NormalTok{long\_testeo, }\AttributeTok{family=}\StringTok{"binomial"}\NormalTok{)}

\DocumentationTok{\#\#\#\#\#\#\#\#\#\#\#\#\#\#\#\#\#\#\#\#\#\#\#\#\#\#\#\#\#\#\#\#\#\#\#\#\#\#\#\#\#\#\#\#\#\#\#\#\#\#\#\#\#\#\#\#\#\#\#\#\#\#\#\#\#\#\#\#\#\#\#\#\#\#\#\#\#\#\#\#}
\DocumentationTok{\#\#\#\#\#\#\#\#\#\#\#\#\#\#\#\#\#\#\#\#\#\#\#\#\#\#\# EVALUACIÓN DE SUPUESTOS \#\#\#\#\#\#\#\#\#\#\#\#\#\#\#\#\#\#\#\#\#\#\#\#\#\#\#\#}
\DocumentationTok{\#\#\#\#\#\#\#\#\#\#\#\#\#\#\#\#\#\#\#\#\#\#\#\#\#\#\#\#\#\#\#\#\#\#\#\#\#\#\#\#\#\#\#\#\#\#\#\#\#\#\#\#\#\#\#\#\#\#\#\#\#\#\#\#\#\#\#\#\#\#\#\#\#\#\#\#\#\#\#\#}

\CommentTok{\# Supuestos m9:}
\DocumentationTok{\#\# Parte fija:}
\NormalTok{sim\_m9 }\OtherTok{\textless{}{-}} \FunctionTok{simulateResiduals}\NormalTok{(m9, }\AttributeTok{n=}\DecValTok{1000}\NormalTok{)}
\FunctionTok{plot}\NormalTok{(sim\_m9)}
\end{Highlighting}
\end{Shaded}

\begin{verbatim}
## Unable to calculate quantile regression for quantile 0.25. Possibly to few (unique) data points / predictions. Will be ommited in plots and significance calculations.
\end{verbatim}

\begin{verbatim}
## Unable to calculate quantile regression for quantile 0.5. Possibly to few (unique) data points / predictions. Will be ommited in plots and significance calculations.
\end{verbatim}

\begin{verbatim}
## Unable to calculate quantile regression for quantile 0.75. Possibly to few (unique) data points / predictions. Will be ommited in plots and significance calculations.
\end{verbatim}

\includegraphics{TP_FINAL_RMarkdown_files/figure-latex/unnamed-chunk-1-4.pdf}

\begin{Shaded}
\begin{Highlighting}[]
\CommentTok{\# Supuestos para variable aleatoria Semana:}
\NormalTok{Bi\_semana }\OtherTok{\textless{}{-}} \FunctionTok{unlist}\NormalTok{(}\FunctionTok{ranef}\NormalTok{(m9))}
\NormalTok{Bi\_semana }\OtherTok{\textless{}{-}}\NormalTok{ Bi\_semana[}\DecValTok{1}\SpecialCharTok{:}\DecValTok{7}\NormalTok{]}
\CommentTok{\# QQPlot con estos residuos:}
\NormalTok{car}\SpecialCharTok{::}\FunctionTok{qqPlot}\NormalTok{(Bi\_semana)}
\end{Highlighting}
\end{Shaded}

\includegraphics{TP_FINAL_RMarkdown_files/figure-latex/unnamed-chunk-1-5.pdf}

\begin{verbatim}
## cond.SEMANA.(Intercept)7 cond.SEMANA.(Intercept)4 
##                        7                        4
\end{verbatim}

\begin{Shaded}
\begin{Highlighting}[]
\CommentTok{\# Prueba de shapiro:}
\FunctionTok{shapiro.test}\NormalTok{(Bi\_semana)}
\end{Highlighting}
\end{Shaded}

\begin{verbatim}
## 
##  Shapiro-Wilk normality test
## 
## data:  Bi_semana
## W = 0.82181, p-value = 0.06684
\end{verbatim}

\begin{Shaded}
\begin{Highlighting}[]
\CommentTok{\# Supuestos para variable aleatoria ID:}
\NormalTok{Bi\_ID }\OtherTok{\textless{}{-}} \FunctionTok{unlist}\NormalTok{(}\FunctionTok{ranef}\NormalTok{(m9))}
\NormalTok{Bi\_ID }\OtherTok{\textless{}{-}}\NormalTok{ Bi\_ID[}\DecValTok{8}\SpecialCharTok{:}\DecValTok{139}\NormalTok{]}
\CommentTok{\# QQPlot con estos residuos:}
\CommentTok{\# QQPlot con estos residuos:}
\NormalTok{car}\SpecialCharTok{::}\FunctionTok{qqPlot}\NormalTok{(Bi\_ID)}
\end{Highlighting}
\end{Shaded}

\includegraphics{TP_FINAL_RMarkdown_files/figure-latex/unnamed-chunk-1-6.pdf}

\begin{verbatim}
##  cond.ID.(Intercept)87 cond.ID.(Intercept)131 
##                     87                    131
\end{verbatim}

\begin{Shaded}
\begin{Highlighting}[]
\CommentTok{\# Prueba de shapiro:}
\FunctionTok{shapiro.test}\NormalTok{(Bi\_ID)}
\end{Highlighting}
\end{Shaded}

\begin{verbatim}
## 
##  Shapiro-Wilk normality test
## 
## data:  Bi_ID
## W = 0.96898, p-value = 0.004093
\end{verbatim}

\begin{Shaded}
\begin{Highlighting}[]
\CommentTok{\# Supuestos m10:}
\DocumentationTok{\#\# Parte fija:}
\NormalTok{sim\_m10 }\OtherTok{\textless{}{-}} \FunctionTok{simulateResiduals}\NormalTok{(m9, }\AttributeTok{n=}\DecValTok{1000}\NormalTok{)}
\FunctionTok{plot}\NormalTok{(sim\_m10)}
\end{Highlighting}
\end{Shaded}

\begin{verbatim}
## Unable to calculate quantile regression for quantile 0.25. Possibly to few (unique) data points / predictions. Will be ommited in plots and significance calculations.
\end{verbatim}

\begin{verbatim}
## Unable to calculate quantile regression for quantile 0.5. Possibly to few (unique) data points / predictions. Will be ommited in plots and significance calculations.
\end{verbatim}

\begin{verbatim}
## Unable to calculate quantile regression for quantile 0.75. Possibly to few (unique) data points / predictions. Will be ommited in plots and significance calculations.
\end{verbatim}

\includegraphics{TP_FINAL_RMarkdown_files/figure-latex/unnamed-chunk-1-7.pdf}

\begin{Shaded}
\begin{Highlighting}[]
\CommentTok{\# Supuestos para variable aleatoria ID:}
\NormalTok{Bi\_ID\_10 }\OtherTok{\textless{}{-}} \FunctionTok{unlist}\NormalTok{(}\FunctionTok{ranef}\NormalTok{(m10))}
\NormalTok{car}\SpecialCharTok{::}\FunctionTok{qqPlot}\NormalTok{(Bi\_ID\_10)}
\end{Highlighting}
\end{Shaded}

\includegraphics{TP_FINAL_RMarkdown_files/figure-latex/unnamed-chunk-1-8.pdf}

\begin{verbatim}
## cond.ID.(Intercept)131  cond.ID.(Intercept)87 
##                    131                     87
\end{verbatim}

\begin{Shaded}
\begin{Highlighting}[]
\FunctionTok{shapiro.test}\NormalTok{(Bi\_ID\_10)}
\end{Highlighting}
\end{Shaded}

\begin{verbatim}
## 
##  Shapiro-Wilk normality test
## 
## data:  Bi_ID_10
## W = 0.98393, p-value = 0.1224
\end{verbatim}

\begin{Shaded}
\begin{Highlighting}[]
\CommentTok{\# PARA M10 (SEMANA FIJA) NO HAY EVIDENCIAS PARA RECHAZAR CUMPLIMIENTO DE SUPUESTOS DE LA PARTE FIJA NI ALEATORIA (ID).}
\CommentTok{\# PARA M9 NO SE CUMPLE LA NORMALIDAD DE LA VARIABLE ALEATORIA ID.}

\FunctionTok{AIC}\NormalTok{(m9,m10)}
\end{Highlighting}
\end{Shaded}

\begin{verbatim}
##     df      AIC
## m9  14 473.4523
## m10 19 462.2870
\end{verbatim}

\begin{Shaded}
\begin{Highlighting}[]
\CommentTok{\# Elegimos usar m10 porque mejora el AIC y se cumplen los supuestos.}
    
\DocumentationTok{\#\#\#\#\#\#\#\#\#\#\#\#\#\#\#\#\#\#\#\#\#\#\#\#\#\#\#\#\#\#\#\#\#\#\#\#\#\#\#\#\#\#\#\#\#\#\#\#\#\#\#\#\#\#\#\#\#\#\#\#\#\#\#\#\#\#\#\#\#\#\#\#\#\#\#\#\#\#\#\#}
\DocumentationTok{\#\#\#\#\#\#\#\#\#\#\#\#\#\#\#\#\#\#\#\#\#\#\#\#\#\# ESTIMACIÓN E INFERENCIA \#\#\#\#\#\#\#\#\#\#\#\#\#\#\#\#\#\#\#\#\#\#\#\#\#\#\#\#\#}
\DocumentationTok{\#\#\#\#\#\#\#\#\#\#\#\#\#\#\#\#\#\#\#\#\#\#\#\#\#\#\#\#\#\#\#\#\#\#\#\#\#\#\#\#\#\#\#\#\#\#\#\#\#\#\#\#\#\#\#\#\#\#\#\#\#\#\#\#\#\#\#\#\#\#\#\#\#\#\#\#\#\#\#\#}

\FunctionTok{Anova}\NormalTok{(m10)}
\end{Highlighting}
\end{Shaded}

\begin{verbatim}
## Analysis of Deviance Table (Type II Wald chisquare tests)
## 
## Response: rta
##                             Chisq Df Pr(>Chisq)    
## TRATAMIENTO                3.3627  3  0.3390115    
## tiempo_testeo              2.0484  2  0.3590747    
## SEMANA                    24.8214  6  0.0003684 ***
## TRATAMIENTO:tiempo_testeo 26.5204  6  0.0001780 ***
## ---
## Signif. codes:  0 '***' 0.001 '**' 0.01 '*' 0.05 '.' 0.1 ' ' 1
\end{verbatim}

\begin{Shaded}
\begin{Highlighting}[]
\FunctionTok{summary}\NormalTok{(m10)}
\end{Highlighting}
\end{Shaded}

\begin{verbatim}
##  Family: binomial  ( logit )
## Formula:          rta ~ TRATAMIENTO * tiempo_testeo + SEMANA + (1 | ID)
## Data: long_testeo
## 
##      AIC      BIC   logLik deviance df.resid 
##    462.3    537.9   -212.1    424.3      377 
## 
## Random effects:
## 
## Conditional model:
##  Groups Name        Variance Std.Dev.
##  ID     (Intercept) 3.151    1.775   
## Number of obs: 396, groups:  ID, 132
## 
## Conditional model:
##                                               Estimate Std. Error z value
## (Intercept)                                  6.775e-01  7.666e-01   0.884
## TRATAMIENTOconstante_bajo                    1.502e+00  8.730e-01   1.720
## TRATAMIENTOcontraste_neg                     1.620e+00  8.651e-01   1.872
## TRATAMIENTOcontraste_pos                     3.911e-01  8.057e-01   0.485
## tiempo_testeo24hs                           -7.442e-07  6.856e-01   0.000
## tiempo_testeo48hs                            4.686e-01  6.876e-01   0.682
## SEMANA2                                     -2.926e-01  8.669e-01  -0.338
## SEMANA3                                     -2.435e-01  8.934e-01  -0.273
## SEMANA4                                      1.253e-01  7.879e-01   0.159
## SEMANA5                                     -1.921e+00  7.759e-01  -2.475
## SEMANA6                                     -1.141e+00  7.062e-01  -1.616
## SEMANA7                                     -4.353e+00  1.016e+00  -4.282
## TRATAMIENTOconstante_bajo:tiempo_testeo24hs -9.648e-01  9.849e-01  -0.980
## TRATAMIENTOcontraste_neg:tiempo_testeo24hs  -1.944e+00  9.866e-01  -1.970
## TRATAMIENTOcontraste_pos:tiempo_testeo24hs   1.455e+00  9.612e-01   1.514
## TRATAMIENTOconstante_bajo:tiempo_testeo48hs -2.152e+00  1.010e+00  -2.131
## TRATAMIENTOcontraste_neg:tiempo_testeo48hs  -4.328e+00  1.115e+00  -3.880
## TRATAMIENTOcontraste_pos:tiempo_testeo48hs   9.861e-01  9.573e-01   1.030
##                                             Pr(>|z|)    
## (Intercept)                                 0.376850    
## TRATAMIENTOconstante_bajo                   0.085358 .  
## TRATAMIENTOcontraste_neg                    0.061162 .  
## TRATAMIENTOcontraste_pos                    0.627387    
## tiempo_testeo24hs                           0.999999    
## tiempo_testeo48hs                           0.495545    
## SEMANA2                                     0.735717    
## SEMANA3                                     0.785213    
## SEMANA4                                     0.873605    
## SEMANA5                                     0.013308 *  
## SEMANA6                                     0.106098    
## SEMANA7                                     1.85e-05 ***
## TRATAMIENTOconstante_bajo:tiempo_testeo24hs 0.327285    
## TRATAMIENTOcontraste_neg:tiempo_testeo24hs  0.048787 *  
## TRATAMIENTOcontraste_pos:tiempo_testeo24hs  0.130136    
## TRATAMIENTOconstante_bajo:tiempo_testeo48hs 0.033061 *  
## TRATAMIENTOcontraste_neg:tiempo_testeo48hs  0.000104 ***
## TRATAMIENTOcontraste_pos:tiempo_testeo48hs  0.302952    
## ---
## Signif. codes:  0 '***' 0.001 '**' 0.01 '*' 0.05 '.' 0.1 ' ' 1
\end{verbatim}

\begin{Shaded}
\begin{Highlighting}[]
\DocumentationTok{\#\#\#\#\#\#\#\#\#\#\#\#\#\#\#\#\#\#\#\#\#\#\#\#\#\#\#\#\#\#\#\#\#\#\#\#\#\#\#\#\#\#\#\#\#\#\#\#\#\#\#\#\#\#\#\#\#\#\#\#\#\#\#\#\#\#\#\#\#\#\#\#\#\#\#\#\#\#\#\#}
\DocumentationTok{\#\#\#\#\#\#\#\#\#\#\#\#\#\#\#\#\#\#\#\#\#\#\#\#\#\#\#\#\#\#\# COMPARACIONES \#\#\#\#\#\#\#\#\#\#\#\#\#\#\#\#\#\#\#\#\#\#\#\#\#\#\#\#\#\#\#\#\#\#}
\DocumentationTok{\#\#\#\#\#\#\#\#\#\#\#\#\#\#\#\#\#\#\#\#\#\#\#\#\#\#\#\#\#\#\#\#\#\#\#\#\#\#\#\#\#\#\#\#\#\#\#\#\#\#\#\#\#\#\#\#\#\#\#\#\#\#\#\#\#\#\#\#\#\#\#\#\#\#\#\#\#\#\#\#}

\CommentTok{\# Seteamos el emmeans:}
\FunctionTok{options}\NormalTok{(}\AttributeTok{emmeans=} \FunctionTok{list}\NormalTok{(}\AttributeTok{emmeans =} \FunctionTok{list}\NormalTok{(}\AttributeTok{infer =} \FunctionTok{c}\NormalTok{(}\ConstantTok{TRUE}\NormalTok{, }\ConstantTok{TRUE}\NormalTok{)),}
                      \AttributeTok{contrast =} \FunctionTok{list}\NormalTok{(}\AttributeTok{infer =} \FunctionTok{c}\NormalTok{(}\ConstantTok{TRUE}\NormalTok{, }\ConstantTok{TRUE}\NormalTok{))))}


\DocumentationTok{\#\#\#\#\# CONTRASTES ORTOGONALES (TENEMOS COMPARACIONES A PRIORI)}

\NormalTok{ortogonales}\OtherTok{\textless{}{-}}\FunctionTok{emmeans}\NormalTok{(m10,}\SpecialCharTok{\textasciitilde{}}\NormalTok{TRATAMIENTO}\SpecialCharTok{*}\NormalTok{tiempo\_testeo,}\AttributeTok{type=}\StringTok{"response"}\NormalTok{,}
                \AttributeTok{contr=}\FunctionTok{list}\NormalTok{(}\StringTok{"3hs\_alto\_pos"}\OtherTok{=}\FunctionTok{c}\NormalTok{(}\DecValTok{1}\NormalTok{,}\DecValTok{0}\NormalTok{,}\DecValTok{0}\NormalTok{,}\SpecialCharTok{{-}}\DecValTok{1}\NormalTok{,}\DecValTok{0}\NormalTok{,}\DecValTok{0}\NormalTok{,}\DecValTok{0}\NormalTok{,}\DecValTok{0}\NormalTok{,}\DecValTok{0}\NormalTok{,}\DecValTok{0}\NormalTok{,}\DecValTok{0}\NormalTok{,}\DecValTok{0}\NormalTok{), }
                     \StringTok{"3hs\_bajo\_neg"}\OtherTok{=}\FunctionTok{c}\NormalTok{(}\DecValTok{0}\NormalTok{,}\DecValTok{1}\NormalTok{,}\SpecialCharTok{{-}}\DecValTok{1}\NormalTok{,}\DecValTok{0}\NormalTok{,}\DecValTok{0}\NormalTok{,}\DecValTok{0}\NormalTok{,}\DecValTok{0}\NormalTok{,}\DecValTok{0}\NormalTok{,}\DecValTok{0}\NormalTok{,}\DecValTok{0}\NormalTok{,}\DecValTok{0}\NormalTok{,}\DecValTok{0}\NormalTok{), }
                     \StringTok{"24hs\_alto\_pos"}\OtherTok{=}\FunctionTok{c}\NormalTok{(}\DecValTok{0}\NormalTok{,}\DecValTok{0}\NormalTok{,}\DecValTok{0}\NormalTok{,}\DecValTok{0}\NormalTok{,}\DecValTok{1}\NormalTok{,}\DecValTok{0}\NormalTok{,}\DecValTok{0}\NormalTok{,}\SpecialCharTok{{-}}\DecValTok{1}\NormalTok{,}\DecValTok{0}\NormalTok{,}\DecValTok{0}\NormalTok{,}\DecValTok{0}\NormalTok{,}\DecValTok{0}\NormalTok{), }
                     \StringTok{"24hs\_bajo\_neg"}\OtherTok{=}\FunctionTok{c}\NormalTok{(}\DecValTok{0}\NormalTok{,}\DecValTok{0}\NormalTok{,}\DecValTok{0}\NormalTok{,}\DecValTok{0}\NormalTok{,}\DecValTok{0}\NormalTok{,}\DecValTok{1}\NormalTok{,}\SpecialCharTok{{-}}\DecValTok{1}\NormalTok{,}\DecValTok{0}\NormalTok{,}\DecValTok{0}\NormalTok{,}\DecValTok{0}\NormalTok{,}\DecValTok{0}\NormalTok{,}\DecValTok{0}\NormalTok{), }
                     \StringTok{"48hs\_alto\_pos"}\OtherTok{=}\FunctionTok{c}\NormalTok{(}\DecValTok{0}\NormalTok{,}\DecValTok{0}\NormalTok{,}\DecValTok{0}\NormalTok{,}\DecValTok{0}\NormalTok{,}\DecValTok{0}\NormalTok{,}\DecValTok{0}\NormalTok{,}\DecValTok{0}\NormalTok{,}\DecValTok{0}\NormalTok{,}\DecValTok{1}\NormalTok{,}\DecValTok{0}\NormalTok{,}\DecValTok{0}\NormalTok{,}\SpecialCharTok{{-}}\DecValTok{1}\NormalTok{), }
                     \StringTok{"48hs\_bajo\_neg"}\OtherTok{=}\FunctionTok{c}\NormalTok{(}\DecValTok{0}\NormalTok{,}\DecValTok{0}\NormalTok{,}\DecValTok{0}\NormalTok{,}\DecValTok{0}\NormalTok{,}\DecValTok{0}\NormalTok{,}\DecValTok{0}\NormalTok{,}\DecValTok{0}\NormalTok{,}\DecValTok{0}\NormalTok{,}\DecValTok{0}\NormalTok{,}\DecValTok{1}\NormalTok{,}\SpecialCharTok{{-}}\DecValTok{1}\NormalTok{,}\DecValTok{0}\NormalTok{)))}

\NormalTok{ortogonales }
\end{Highlighting}
\end{Shaded}

\begin{verbatim}
## $emmeans
##  TRATAMIENTO    tiempo_testeo   prob     SE  df lower.CL upper.CL null t.ratio
##  constante_alto 3hs           0.3916 0.1441 377   0.1638    0.679  0.5  -0.728
##  constante_bajo 3hs           0.7430 0.1193 377   0.4583    0.908  0.5   1.699
##  contraste_neg  3hs           0.7648 0.1097 377   0.4950    0.915  0.5   1.934
##  contraste_pos  3hs           0.4877 0.1379 377   0.2433    0.738  0.5  -0.090
##  constante_alto 24hs          0.3916 0.1441 377   0.1638    0.679  0.5  -0.728
##  constante_bajo 24hs          0.5241 0.1504 377   0.2519    0.783  0.5   0.160
##  contraste_neg  24hs          0.3176 0.1248 377   0.1304    0.591  0.5  -1.328
##  contraste_pos  24hs          0.8030 0.0954 377   0.5545    0.930  0.5   2.329
##  constante_alto 48hs          0.5070 0.1506 377   0.2393    0.771  0.5   0.047
##  constante_bajo 48hs          0.3493 0.1393 377   0.1386    0.642  0.5  -1.015
##  contraste_neg  48hs          0.0642 0.0423 377   0.0168    0.215  0.5  -3.802
##  contraste_pos  48hs          0.8030 0.0954 377   0.5545    0.930  0.5   2.329
##  p.value
##   0.4671
##   0.0902
##   0.0539
##   0.9287
##   0.4671
##   0.8727
##   0.1851
##   0.0204
##   0.9627
##   0.3108
##   0.0002
##   0.0204
## 
## Results are averaged over the levels of: SEMANA 
## Confidence level used: 0.95 
## Intervals are back-transformed from the logit scale 
## Tests are performed on the logit scale 
## 
## $contrasts
##  contrast      odds.ratio    SE  df lower.CL upper.CL null t.ratio p.value
##  3hs_alto_pos       0.676 0.545 377   0.1387    3.297    1  -0.485  0.6277
##  3hs_bajo_neg       0.889 0.767 377   0.1631    4.845    1  -0.137  0.8914
##  24hs_alto_pos      0.158 0.134 377   0.0296    0.842    1  -2.168  0.0308
##  24hs_bajo_neg      2.367 1.980 377   0.4566   12.266    1   1.029  0.3039
##  48hs_alto_pos      0.252 0.213 377   0.0481    1.324    1  -1.634  0.1031
##  48hs_bajo_neg      7.832 7.238 377   1.2726   48.202    1   2.227  0.0265
## 
## Results are averaged over the levels of: SEMANA 
## Confidence level used: 0.95 
## Intervals are back-transformed from the log odds ratio scale 
## Tests are performed on the log odds ratio scale
\end{verbatim}

\begin{Shaded}
\begin{Highlighting}[]
\DocumentationTok{\#\#\#\#\#\#\#\#\#\#\#\#\#\#\#\#\#\#\#\#\#\#\#\#\#\#\#\#\#\#\#\#\#\#\#\#\#\#\#\#\#\#\#\#\#\#\#\#\#\#\#\#\#\#\#\#\#\#\#\#\#\#\#\#\#\#\#\#\#\#\#\#\#\#\#\#\#\#\#\#}
\DocumentationTok{\#\#\#\#\#\#\#\#\#\#\#\#\#\#\#\#\#\#\#\#\#\#\#\#\#\#\#\#\#\#\# GRÁFICO FINAL \#\#\#\#\#\#\#\#\#\#\#\#\#\#\#\#\#\#\#\#\#\#\#\#\#\#\#\#\#\#\#\#\#\#}
\DocumentationTok{\#\#\#\#\#\#\#\#\#\#\#\#\#\#\#\#\#\#\#\#\#\#\#\#\#\#\#\#\#\#\#\#\#\#\#\#\#\#\#\#\#\#\#\#\#\#\#\#\#\#\#\#\#\#\#\#\#\#\#\#\#\#\#\#\#\#\#\#\#\#\#\#\#\#\#\#\#\#\#\#}

\NormalTok{res\_modelo }\OtherTok{\textless{}{-}} \FunctionTok{as.data.frame}\NormalTok{(ortogonales}\SpecialCharTok{$}\NormalTok{emmeans)}

\CommentTok{\# Cambiamos los nombres del tiempo de testeo:}
\NormalTok{res\_modelo}\SpecialCharTok{$}\NormalTok{tiempo\_testeo }\OtherTok{\textless{}{-}} \FunctionTok{with}\NormalTok{(res\_modelo, }\FunctionTok{factor}\NormalTok{(tiempo\_testeo,}
                                                    \AttributeTok{levels =} \FunctionTok{c}\NormalTok{(}\StringTok{"3hs"}\NormalTok{,}\StringTok{"24hs"}\NormalTok{,}\StringTok{"48hs"}\NormalTok{),}
                                                    \AttributeTok{labels =} \FunctionTok{c}\NormalTok{(}\StringTok{"3"}\NormalTok{,}\StringTok{"24"}\NormalTok{,}\StringTok{"48"}\NormalTok{)))}
\CommentTok{\# Reordenamos los levels para la leyenda del grafico:}
\NormalTok{res\_modelo}\SpecialCharTok{$}\NormalTok{TRATAMIENTO }\OtherTok{\textless{}{-}} \FunctionTok{factor}\NormalTok{(res\_modelo}\SpecialCharTok{$}\NormalTok{TRATAMIENTO,}
                                 \AttributeTok{levels =} \FunctionTok{c}\NormalTok{(}\StringTok{"contraste\_pos"}\NormalTok{,}\StringTok{"constante\_alto"}\NormalTok{,}
                                            \StringTok{"constante\_bajo"}\NormalTok{,}\StringTok{"contraste\_neg"}\NormalTok{))}

\CommentTok{\# Graficamos:}
\NormalTok{gp\_final }\OtherTok{\textless{}{-}} \FunctionTok{ggplot}\NormalTok{(res\_modelo,}\FunctionTok{aes}\NormalTok{(}\AttributeTok{x=}\NormalTok{tiempo\_testeo,}\AttributeTok{y=}\NormalTok{prob,}\AttributeTok{group=}\NormalTok{TRATAMIENTO,}\AttributeTok{color=}\NormalTok{TRATAMIENTO)) }\SpecialCharTok{+}  
  \FunctionTok{labs}\NormalTok{(}\AttributeTok{x=}\StringTok{"Tiempo de evaluacion (hs)"}\NormalTok{,}\AttributeTok{y=}\StringTok{"Proporcion de PER"}\NormalTok{,}\AttributeTok{title=}\StringTok{"Estimaciones del modelo"}\NormalTok{) }\SpecialCharTok{+}
  \FunctionTok{geom\_errorbar}\NormalTok{(}\FunctionTok{aes}\NormalTok{(}\AttributeTok{ymin=}\NormalTok{(prob}\SpecialCharTok{{-}}\NormalTok{res\_modelo}\SpecialCharTok{$}\NormalTok{SE),}\AttributeTok{ymax=}\NormalTok{(prob}\SpecialCharTok{+}\NormalTok{res\_modelo}\SpecialCharTok{$}\NormalTok{SE)),}
                \AttributeTok{width=}\NormalTok{.}\DecValTok{2}\NormalTok{,}\AttributeTok{color=}\StringTok{"lightgrey"}\NormalTok{) }\SpecialCharTok{+}
  \FunctionTok{geom\_line}\NormalTok{(}\AttributeTok{linetype=}\DecValTok{2}\NormalTok{) }\SpecialCharTok{+}
  \FunctionTok{geom\_point}\NormalTok{(}\AttributeTok{size=}\DecValTok{2}\NormalTok{,}\AttributeTok{shape=}\FunctionTok{c}\NormalTok{(}\DecValTok{15}\NormalTok{,}\DecValTok{16}\NormalTok{,}\DecValTok{16}\NormalTok{,}\DecValTok{15}\NormalTok{,}\DecValTok{15}\NormalTok{,}\DecValTok{16}\NormalTok{,}\DecValTok{16}\NormalTok{,}\DecValTok{15}\NormalTok{,}\DecValTok{15}\NormalTok{,}\DecValTok{16}\NormalTok{,}\DecValTok{16}\NormalTok{,}\DecValTok{15}\NormalTok{)) }\SpecialCharTok{+} \CommentTok{\#shape puntos}
  \FunctionTok{labs}\NormalTok{(}\AttributeTok{col=}\StringTok{"Tratamiento"}\NormalTok{) }\SpecialCharTok{+}
  \FunctionTok{theme\_classic}\NormalTok{() }\SpecialCharTok{+}
  \FunctionTok{ylim}\NormalTok{(}\DecValTok{0}\NormalTok{,}\DecValTok{1}\NormalTok{) }\SpecialCharTok{+}
  \FunctionTok{scale\_colour\_manual}\NormalTok{(}\AttributeTok{labels =} \FunctionTok{c}\NormalTok{(}\StringTok{"Contraste positivo"}\NormalTok{,}\StringTok{"Constante alto"}\NormalTok{,}
                                 \StringTok{"Constante bajo"}\NormalTok{,}\StringTok{"Contraste negativo"}\NormalTok{),}
                      \AttributeTok{values=}\FunctionTok{c}\NormalTok{(}\StringTok{"\#00b050"}\NormalTok{,}\StringTok{"\#70ad47"}\NormalTok{,}\StringTok{"\#ed7c31"}\NormalTok{,}\StringTok{"\#ff0000"}\NormalTok{)) }\SpecialCharTok{+} \CommentTok{\#leyenda}
  \FunctionTok{guides}\NormalTok{(}\AttributeTok{color =} \FunctionTok{guide\_legend}\NormalTok{(}\AttributeTok{override.aes=}\FunctionTok{list}\NormalTok{(}\AttributeTok{shape=}\FunctionTok{c}\NormalTok{(}\DecValTok{15}\NormalTok{,}\DecValTok{15}\NormalTok{,}\DecValTok{16}\NormalTok{,}\DecValTok{16}\NormalTok{)))) }\SpecialCharTok{+}  \CommentTok{\#leyenda}
  \FunctionTok{annotate}\NormalTok{(}\AttributeTok{geom=}\StringTok{"text"}\NormalTok{,}\AttributeTok{x=}\StringTok{"24"}\NormalTok{,}\AttributeTok{y=}\DecValTok{1}\NormalTok{,}\AttributeTok{label=}\StringTok{"*"}\NormalTok{,}\AttributeTok{color=}\StringTok{"\#00b050"}\NormalTok{,}\AttributeTok{size=}\DecValTok{10}\NormalTok{) }\SpecialCharTok{+} \CommentTok{\#significancia}
  \FunctionTok{annotate}\NormalTok{(}\AttributeTok{geom=}\StringTok{"text"}\NormalTok{,}\AttributeTok{x=}\StringTok{"48"}\NormalTok{,}\AttributeTok{y=}\DecValTok{1}\NormalTok{,}\AttributeTok{label=}\StringTok{"*"}\NormalTok{,}\AttributeTok{color=}\StringTok{"\#ff0000"}\NormalTok{,}\AttributeTok{size=}\DecValTok{10}\NormalTok{)   }\CommentTok{\#significancia}
  
\NormalTok{gp\_final}
\end{Highlighting}
\end{Shaded}

\includegraphics{TP_FINAL_RMarkdown_files/figure-latex/unnamed-chunk-1-9.pdf}

\begin{Shaded}
\begin{Highlighting}[]
\DocumentationTok{\#\#\#\#\#\#\#\#\#\# RESUMEN DE GRAFICOS \#\#\#\#\#\#\#\#\#\# }
\CommentTok{\# Descriptiva entrenamiento:}
\NormalTok{gp\_entr}
\end{Highlighting}
\end{Shaded}

\includegraphics{TP_FINAL_RMarkdown_files/figure-latex/unnamed-chunk-1-10.pdf}

\begin{Shaded}
\begin{Highlighting}[]
\CommentTok{\# Descriptiva testeo:}
\NormalTok{gp\_testeo}
\end{Highlighting}
\end{Shaded}

\includegraphics{TP_FINAL_RMarkdown_files/figure-latex/unnamed-chunk-1-11.pdf}

\begin{Shaded}
\begin{Highlighting}[]
\CommentTok{\# Semana como covariable:}
\NormalTok{gp\_semana}
\end{Highlighting}
\end{Shaded}

\includegraphics{TP_FINAL_RMarkdown_files/figure-latex/unnamed-chunk-1-12.pdf}

\begin{Shaded}
\begin{Highlighting}[]
\CommentTok{\# Grafico final con significancias:}
\NormalTok{gp\_final}
\end{Highlighting}
\end{Shaded}

\includegraphics{TP_FINAL_RMarkdown_files/figure-latex/unnamed-chunk-1-13.pdf}

\end{document}
